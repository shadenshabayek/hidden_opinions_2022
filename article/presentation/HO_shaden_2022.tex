\documentclass[xcolor=table]{beamer}
%\documentclass[xcolor=table,handout]{beamer}
%\documentclass[usenames,dvipsnames]{beamer}
%
% Choose how your presentation looks.
%
% For more themes, color themes and font themes, see:
% http://deic.uab.es/~iblanes/beamer_gallery/index_by_theme.html
%

\mode<presentation>
{
  \usetheme{default}      % or try Darmstadt, Madrid, Warsaw, ...
%%  \usecolortheme{default} % or try albatross, beaver, crane, ...
 %% \usefonttheme{default}  % or try serif, structurebold, ...
  %%\setbeamertemplate{navigation symbols}{}
  %%\setbeamertemplate{caption}[numbered]
} 

%%
\makeatletter

\definecolor{beamer@blendedblue}{rgb}{0.68, 0.16, 0.16} % changed this

\setbeamercolor{normal text}{fg=black,bg=white}
\setbeamercolor{alerted text}{fg=red}
\setbeamercolor{example text}{fg=red!50!black}

\setbeamercolor{structure}{fg=beamer@blendedblue}

\setbeamercolor{background canvas}{parent=normal text}
\setbeamercolor{background}{parent=background canvas}

\setbeamercolor{palette primary}{fg=red,bg=red} % changed this
\setbeamercolor{palette secondary}{use=structure,fg=structure.fg!100!red} % changed this
\setbeamercolor{palette tertiary}{use=structure,fg=structure.fg!100!red} % changed this
\setbeamercovered{invisible}
\makeatother

%%%
\usepackage[english]{babel}
\usepackage[utf8x]{inputenc}
\usepackage[most]{tcolorbox}
\usepackage{amsthm}
\usepackage{graphics}
\usepackage{tikz,lipsum,lmodern}
\usepackage{combelow}
\usepackage{bbm}
\usepackage{newunicodechar}
\newunicodechar{?}{\cb{s}}
\usepackage{mathtools}
\usepackage{tabularx}
\usepackage{blindtext}
\usepackage{amsmath}
\usepackage[outline]{contour}
\usepackage{tikz}
%\usepackage{fourier}
\useinnertheme{rectangles}

\usepackage{url}
\usepackage{background}
\usepackage{lastpage}
\usepackage{graphbox}

\logo{\pgfputat{\pgfxy(-11.4,-0.4)}{\pgfbox[center,base]{\includegraphics[scale=0.28]{./img/logo.png}}}
}
\setbeamertemplate{footline}[frame number]

\title[Hidden Opinions]{Hidden Opinions}
\author{Shaden Shabayek}
\institute{ Post-doctoral Researcher \\ Sciences Po Medialab}
\date{CTN Barcelona \\ May 20, 2022 }

\makeatletter
\newbox\@backgroundblock
\newenvironment{backgroundblock}[2]{%
  \global\setbox\@backgroundblock=\vbox\bgroup%
    \unvbox\@backgroundblock%
    \vbox to0pt\bgroup\vskip#2\hbox to0pt\bgroup\hskip#1\relax%
}{\egroup\egroup\egroup}
\addtobeamertemplate{background}{\box\@backgroundblock}{}
\makeatother

\begin{document}
\begin{frame}[noframenumbering]

 \titlepage
 \thispagestyle{empty}
  

\end{frame}

\begin{frame}{Motivation}

%\begin{backgroundblock}{55mm}{55mm}
%\includegraphics[width=50mm]{./img/tw.png}
%\end{backgroundblock}

\begin{itemize} \setlength\itemsep{1em}
	\item Group {\bf\color{purple}Polarization}   \smallskip 
	
		\begin{itemize} \setlength\itemsep{1em}
			\item[$\star$] Empirical studies, regularity: high internal consensus and sharp disagreement across groups. Examples: climate catastrophe... 
			\item[$\star$] Implications: economic, political, institutional... also can be a driver of radicalization, cultural shifts, etc. 
			\item[$\star$] Role  of Digital platforms? Online Social networks? 
			\item[$\star$] Is it the {\bf\color{purple}technology} (e.g. echo chambers, algorithms) ? Is it the {\bf\color{purple}actors} themselves? 
		\end{itemize}
	\item Here, {\bf\color{purple}focus of one possible driver} \& model it: 
		\begin{itemize} \setlength\itemsep{1em}
			\item[$\star$] Inequality of attention towards expressed opinions. (mix of the technology argument and the individual behavior argument)
		\end{itemize}
\end{itemize}
\end{frame}
%%%%%%
\begin{frame}{Motivation \& Idea of the model}
\begin{itemize} \setlength\itemsep{1em}
	\item Here, focus on one possible driver \& model it: 
		\begin{itemize} \setlength\itemsep{1em}
			\item[$\star$] Inequality of attention towards expressed opinions. 
		\end{itemize}
	\item How ? refine DeGroot...
		\begin{itemize}  \setlength\itemsep{1em}
			\item[$\star$] Introduce an {\bf\color{purple}Expression heuristic}
			\item[$\star$] Express  \begin{itemize} \item[$\circ$] Influential individuals {\bf\color{purple}(high centrality)} take part of the discussion and can interact with other influential individuals: {\bf\color{purple}like-minded} or {\bf\color{purple}ideologically-opposed}. \end{itemize}
			\item[$\star$] Hide \begin{itemize} \item[$\circ$] Normal individuals {\bf\color{purple}(low centrality)} will have little impact on the conversation (update \`{a} la Degroot): e.g. {\bf\color{purple}Retweet} count  \end{itemize}
		\end{itemize}
\end{itemize}
\end{frame}
%%%%%%
\begin{frame}{Literature: models of opinion dynamics}

		\begin{itemize} \setlength\itemsep{1em}
			\item[$\star$] Large literature about {\bf\color{purple}consensus} or agreement
				\begin{itemize} \smallskip
				
					\item[$\circ$] take the average of opinions expressed by all direct contacts, see DeGroot (1974), Harary (1959), French (1956)...
				\end{itemize} 
			\item[$\star$] Strand about {\bf\color{purple}disagreement} \smallskip
				\begin{itemize} \setlength\itemsep{1em} 
				
					\item[$\circ$]  {\bf\color{purple}Stubborn agents}, see Friedkin(2015), Acemoglu et al (2013), Yildiz et al. (2013), Sadler (2019). %introduce stubborn agents: in a voter model set-up where opinions are discrete and can take only two discrete real values either $a$ or$ b$. %: either stubborn (never update opinion), or update \`{a} la Degroot. 
					\item[$\circ$] {\bf\color{purple}Bounded confidence interval}, see for example, Hegselmann and Krauss (2002), Jager and Amblard (2005). %:  update opinion by taking an average over the opinions of neighbors whose opinion difference falls within a confidence interval.
				\end{itemize}
		
		\item[$\star$] But ... (social psychology \& sociology) 
			\begin{itemize}
				\item[$\circ$] Not all individuals are equal... influence matters. 
				\item[$\circ$] Agents may not always share their opinions with others.  
			\end{itemize}
		\end{itemize}
\end{frame}
%
\begin{frame}{Literature: social psychology}
\begin{itemize}
	\item Stasser and Titus (1985) : individuals in social contexts, do not always share their own opinion or the information they hold.
		\begin{itemize} \setlength\itemsep{1em}
			%\item[$\star$] communication: corrective function, members have incomplete information but together gather the different pieces of the puzzle.
			\item[{\color{gray}$\star$}] {\color{gray}Political set-up: elect Best, Okay or Ohum. }
			%\item[$\star$] Experiment? distribute different subset of desirable traits of Best and a different subset of Okay’s undesirable traits over the members of the group, such that from each one’s individual perspective Okay appeared more positive than Best. 
			\item[{\color{gray}$\star$}] {\color{gray}The group as a whole had complete (but dispersed) information about Best, from each one’s individual perspective Okay appeared more positive than Best. }%they could exchange it and come to the conclusion that Best was actually the best candidate. 
			\item[{\color{gray}$\star$}] {\color{gray}Before AND after discussion Best received 25\% of endorsement.} 
		\end{itemize}
		\item Larson et al. (1996): sharing a unique piece of information during a discussion, is more likely by higher status members rather than lower status members.
			\begin{itemize} \setlength\itemsep{1em}
				\item[{\color{gray}$\star$}] {\color{gray}Experiment: residents, interns \& 3rd-year medical students.}
				\item[{\color{gray}$\star$}] {\color{gray}Main result: residents were more likely to repeat (unique) information when compared to interns and students.}
			\end{itemize} 	
\end{itemize}
%repeating a unique piece of information, leading to the formation of group opinion during a discussion, is more likely by higher status members (experts, leaders, etc.) rather than lower status members.
\end{frame}
%
%Larson et al. (1996) [17] suggest that repeating a unique piece of information, leading to the formation of group opinion during a discussion, is more likely by higher status members (experts, leaders, etc.) rather than lower status members. They ran an experiment with residents, interns and 3rd-year medical students and they show that residents were more likely to repeat (unique) information when compared to interns and students.


\begin{frame}[noframenumbering, plain]
\begin{tcolorbox}[enhanced,attach boxed title to top center={yshift=-3mm,yshifttext=-1mm}, colback=red!3,colframe=red!40,colbacktitle=red!40 ,fonttitle=\bfseries, boxed title style={size=small,colframe=red!50} ] 
\centering Model
\end{tcolorbox}
\end{frame}

\begin{frame}{Model}
\begin{itemize}\setlength\itemsep{1em}
	\item {\bf\color{purple}Network}, {\bf\color{purple}Influence} \& {\bf\color{purple}Initial Opinions} 
	\begin{itemize}\setlength\itemsep{1em}
		\item[$\star$] Group of individuals $N = \{ 1, \ldots, n\}$ embedded in
		\item[$\star$] ... a connected symmetric network: $G$, $g_{ij} = g_{ji} \in \{ 0,1\}$.
		\item[$\star$] Set of friends of $i \in N$: $\{ j \in N: g_{ij} =1 \}$ and $d_i = |N_i|$. 
		\item[$\star$] The influence or {\bf\color{purple}centrality} of $i \in N$: $\lambda_i (G) \in \mathbb{R}_{+}^{*}$.
		\item[$\star$] Exogenous initial opinion for $i \in N$: $\alpha_{i,0} \in [-1,1]$
		
	\end{itemize}
	\end{itemize}
	\begin{tcolorbox}[enhanced,attach boxed title to top center={yshift=-3mm,yshifttext=-1mm}, colback=red!3,colframe=red!40,colbacktitle=red!40 ,fonttitle=\bfseries, boxed title style={size=small,colframe=red!50}, title ={Definition} ]  At period $t \geq 0$, for $\tau \in (0,1)$, $i \neq j \in N$ are:
	\begin{itemize}
		\item {\bf\color{purple} like-minded}: $|\alpha_{i,t} - \alpha_{j,t}| < \tau$
		\item {\bf\color{purple} ideologically-opposed}: $|\alpha_{i,t} - \alpha_{j,t}| \geq \tau$
	\end{itemize}
	\end{tcolorbox}

\end{frame}
%
\begin{frame}{Model}
	%\begin{itemize}
		%\item For $i \in N$, with centrality/influence $\delta_i (G)$
		
				\begin{tcolorbox}[enhanced,attach boxed title to top center={yshift=-3mm,yshifttext=-1mm}, colback=red!3,colframe=red!40,colbacktitle=red!40 ,fonttitle=\bfseries, boxed title style={size=small,colframe=red!50}, title ={Opinion Updating} ]  For $i \in N$, with centrality $\delta_i (G)$
\begin{equation*} 
	\begin{cases}
		 \text{ Hide (rule 1): DeGroot } & \text{ if } \delta_i (G) < \delta^{*} (G) \\
		\text{ Express (rule 2): see Below } & \text{ if } \delta_i (G) \geq \delta^{*} (G)
	\end{cases}
\end{equation*} 
	\end{tcolorbox}
	
	\begin{itemize}
		\item Rule 2 \begin{itemize} \item[$\star$] $\forall j \in N_i$, such that $ \delta_j(G) \geq \delta^{*} (G)$ \item[$\star$] {$\mu \in (0,1/2)$} and {\bf\color{purple}$\alpha_{i,t} \in [-1,1]$} \end{itemize} 
	\end{itemize}
	%\smallskip
 	\begin{equation*}
	\begin{split}
	 \alpha_{i,{t}}  =  & \; \alpha_{i,{t-1}} \\  +   & \; \mu \big( \underbrace{\sum_{j \in \underline{N}_{i,t-1}}   (  \alpha_{j,t-1} - \alpha_{i,t-1} )}_{\substack{\text{{\bf\color{purple}Attractive Effect}} \\ \\ \text{{\bf\color{purple}Like-minded} neighbors } \underline{N}_{i,t}}} \; - \underbrace{\sum_{j \in \overline{N}_{i,t-1}} ( \alpha_{j,t-1} - \alpha_{i,t-1} ) \big)}_{\substack{\text{{\bf\color{purple}Repulsive Effect}} \\ \\ \text{{\bf\color{purple}Ideologically-opposed} neighbors } \overline{N}_{i,t}}}    \label{lawofmotion}
	 \end{split}
 	\end{equation*}

%$\text{ s.t. } \alpha_{i,t} \in [-1,1]$ \smallskip


\end{frame}
%
\begin{frame}{How does Rule 2 work? some examples! }
\begin{itemize} \item[$\star$] Hypo: threshold opinion difference $\tau = 0.5$ \& same influence \end{itemize}
\begin{figure}
\centering
\begin{minipage}{.5\textwidth}
  \centering
  \includegraphics[scale = 0.55]{./img/ideo_opp_1.png}
  %\captionof{figure}{A figure}
  %\label{fig:test1}
\end{minipage}%
\begin{minipage}{.5\textwidth}
  \centering
  \includegraphics[scale = 0.55]{./img/plot_ideo_opposed.jpg}
 % \captionof{figure}{Another figure}
  %\label{fig:test2}
\end{minipage}
\end{figure}
%
\begin{figure}
\centering
\begin{minipage}{.5\textwidth}
  \centering
  \includegraphics[scale = 0.55]{./img/cons_2.png}
  %\captionof{figure}{A figure}
  %\label{fig:test1}
\end{minipage}%
\begin{minipage}{.5\textwidth}
  \centering
  \includegraphics[scale = 0.55]{./img/plot_like_minded.jpg}
 % \captionof{figure}{Another figure}
  %\label{fig:test2}
\end{minipage}
\end{figure}
\end{frame}
%
\begin{frame}{How does Rule 2 work? more examples! }
\begin{itemize} \item[$\star$] Hypo: threshold opinion difference $\tau = 0.5$ \& same influence \end{itemize}
\begin{figure}
\centering
\begin{minipage}{.5\textwidth}
  \centering
  \includegraphics[scale = 0.55]{./img/moderate.png}
  %\captionof{figure}{A figure}
  %\label{fig:test1}
\end{minipage}%
\begin{minipage}{.5\textwidth}
  \centering
  \includegraphics[scale = 0.55]{./img/plot_moderate.jpg}
 %\caption{figure}{Moderate Opinions}
  %\label{fig:test2}
\end{minipage}
\end{figure}
%
\begin{figure}
\centering
\begin{minipage}{.5\textwidth}
  \centering
  \includegraphics[scale = 0.55]{./img/ideo_non_monotonic.png}
  %\captionof{figure}{A figure}
  %\label{fig:test1}
\end{minipage}%
\begin{minipage}{.5\textwidth}
  \centering
  \includegraphics[scale = 0.55]{./img/plot_ideo_opposed_non_monotonic.jpg}
 % \captionof{figure}{Another figure}
  %\label{fig:test2}
\end{minipage}
\end{figure}
\end{frame}
%
\begin{frame}{Results? Long-run opinions? Hearing matrix}

\begin{itemize}\setlength\itemsep{1em}
	\item Opinions of ``expressers" (influential players) and the way they are connected governs the dynamics of opinions...
\end{itemize}
\begin{tcolorbox}[enhanced,attach boxed title to top center={yshift=-3mm,yshifttext=-1mm}, colback=red!3,colframe=red!40,colbacktitle=red!40, title=Proposition ,fonttitle=\bfseries, boxed title style={size=small,colframe=red!50} ] 

%Let $i \in N$, s.t. $\delta_i \geq1/2$ .
%\eqref{lawofmotion}. 
\begin{itemize}
\setlength\itemsep{1em}
%	\item[$(i)$] <1->  If player $i$ has no neighbors who express then %$i$ is forever stubborn.  
%%	$\forall j \in N_i \cap C \mbox{ }  \Rightarrow \mbox{ } \forall t \geq 1, $ 
%$a_{i,t} = \alpha_{i,0}$
	\item[$(i)$]  If $i$ has exactly $k$ neighbors who $express$ and they are all {\bf\color{purple}like-minded}, then $\exists t^* \geq t, \mbox{ }\forall t' \geq t^* $ :
	%$ \exists! j \in N_i \cap E,  |\alpha_{i,0} - \alpha_{j,0} |< \tau \Rightarrow \exists T \geq t, \mbox{ }\forall t' \geq T, $ 
	$ a_{i,t'} = ({\alpha_{i,0} + \alpha_{k_1,0}  + \ldots + \alpha_{k_k,0} })/{k}$
	\item[$(ii)$]  If player $i$ has, at every period, {\bf\color{purple}at least one neighbor} $j$ who expresses and {\bf\color{purple}ideologically opposed}, then $ \exists t^{*} \geq t, \mbox{ }\forall t' \geq t^{*},$ $ \mbox{ } a_{i,t'} = -1, \mbox{ }a_{j,t'} = 1$
	\item[$(iii)$]  If player $i$ has, at every period, {\bf\color{purple}like-minded neighbors} who express and is {\bf\color{purple}linked by a path} to a pair of ideologically opposed expressers, then  $a_{i,t} = \alpha_{i,0}$
\end{itemize}
\end{tcolorbox}
\end{frame}
%
\begin{frame}{Result 1: convergece}

\end{frame}
%
\begin{frame}{Result 2}

\end{frame}
%
\begin{frame}[noframenumbering, plain]
\begin{tcolorbox}[enhanced,attach boxed title to top center={yshift=-3mm,yshifttext=-1mm}, colback=red!3,colframe=red!40,colbacktitle=red!40 ,fonttitle=\bfseries, boxed title style={size=small,colframe=red!50} ] 
\centering Simulations
\end{tcolorbox}
\end{frame}
%%%%%%
\begin{frame}{Simulations}
\begin{itemize} \item
\end{itemize}
\end{frame}

\end{document}